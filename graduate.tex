\documentclass[a4paper]{jsarticle}
\usepackage[dvipdfmx]{graphicx} % Required for inserting images
\usepackage{amsmath}
\usepackage{amsthm}
\usepackage{amssymb}
\usepackage[margin=20truemm]{geometry}
\usepackage{graphicx}
\usepackage{enumerate}
\usepackage{tikz-cd}
 \theoremstyle{definition}
    \newtheorem{dfn}{定義}[section]
    \newtheorem{prop}[dfn]{命題}
    \newtheorem{lem}[dfn]{補題}
    \newtheorem{thm}[dfn]{定理}
    \newtheorem{cor}[dfn]{系}
    \newtheorem{exam}[dfn]{例}
    \newtheorem{rem}[dfn]{注意}
    \newtheorem{hsk}[dfn]{補足}
    \renewcommand\proofname{\bf 証明}

    \renewcommand{\qedsymbol}{$\blacksquare$}
\newcommand{\SimpComp}{{\mathrm{SimpComp}}}
\newcommand{\Fun}[2]{[#1,~#2]}
\newcommand{\Vect}{{\mathrm{Vect}}}
\newcommand{\SimpCompInc}{{\mathrm{SimpCompInc}}}
\newcommand{\grmodZ}{{\mathrm{grmod \mathbb{Z}_2[x]}}}
\newcommand{\Hom}{{\mathrm{Hom}}}


\title{graduate book}
\author{猪原 盛寿}
\date{October 19th 2023}


\begin{document}
\Large
\maketitle

\section{導入 Introduction}
良心に恥じぬということだけが、我々の確かな報酬である。\\
---セオドア・ソレンソン\\
勇敢にもこの論文を手に取った全ての人に感謝を込めて
\section{背景 background}

\begin{dfn}
圏(category)$X$とは以下の条件を満たす数学的構造である. 
\begin{itemize}
    \item 対象(object)の集まり ob($X$)
    \item 射(morphism)の集まり Hom($A,B$)
    (各対象$A, B$に対して$A$から$B$への射の集まりをHom($A, B$)とする)
    \item 射の合成と呼ばれる関数$(\circ)$
\end{itemize}
\begin{equation}
    \begin{array}{lllll}
     \circ &: \Hom (B, C) &\times \Hom (A, B) &\rightarrow &\Hom (A, C)  \\
         &    & (g,f) & \mapsto&  g\circ f
\end{array}
\end{equation}
から構成され, 射は次の二つの法則を満たす.
\begin{enumerate}[(1)]
    \item 結合法則. 任意の射$f\in \Hom(A,B), g\in \Hom(B,C), h\in \Hom(C,D),$は次の等式を満たす. $(h \circ g) \circ f = h \circ (g \circ f)$  
    \item 単位法則. 任意の対象$A$に対し, 恒等射と呼ばれる射$1_A:A\rightarrow A$が存在し, 次の等式を満たす.  $f\circ 1_A = 1_B\circ f = f$  
\end{enumerate}
\begin{equation}
    \begin{tikzcd}
    A\ar{r}{f} \ar[loop left,"1_A"] \ar[bend right=30,swap]{rr}{g\circ f}& B\ar{r}{g} \ar[bend left=30]{rr}{h\circ g}& C\ar{r}{h} & D
\end{tikzcd}
\end{equation}
\end{dfn}

\begin{hsk}
まとめると圏とは, 対象, 射, 射の合成からなり, 射が二つの法則を満たすものである.
対象とは, 集合でいうところの要素のこと. A, ☆, 1, 〇, ...など集合や数字に限らない.
射とは, 対象間に走る矢印のことで.対象同士を結ぶこと, 向きが存在すること, が条件となる. $A \rightarrow B$, 〇$\rightarrow$☆, などで$A$と$B$の間の射は一本とは限らない. このときのA, 〇を始域といい, 反対にB, ☆は終域という. 圏論が「関係性」の数学と呼ばれるように, 圏では主に射の方に着目する. 構成要素の一つに射の合成が入っている. 片方の終域ともう片方の始域が一致するとき, 二つの射を合成した射が一つ存在しなければならない. 圏$X$が$A\xrightarrow[]{a} B\xrightarrow[]{b} C$かつ, $A\xrightarrow[]{c} C$のとき, $c=b\circ a$である. 結合法則や単位法則は, 写像のそれと同様である. 
\\
\end{hsk}

\begin{exam}
対象を1,2,...の自然数全体, 射を$m\leqq n$ならば$m \rightarrow n$ ($m, n \in \mathbb{N}$)とする集まり$C$を考える.\\ ob($C$)$= \mathbb{N}$. $m\leqq n$のとき$\Hom (m, n)=\{m\rightarrow n\}$. $m>n$のとき$\Hom (m, n)=\{\}$. $l, m, n\in$ ob$(X)$, $l \rightarrow m \rightarrow n$ならば, $l\leqq m\leqq n$から$l\leqq n$より$l \rightarrow n$. よって射の合成は含まれている. では射が二つの法則を満たすか確認する. 
 \begin{enumerate}[(1)]
        \item $k\rightarrow l\rightarrow m\rightarrow n$ならば, 合成により$k\rightarrow m$, $l \rightarrow n$が存在し, さらに二つの射を利用し同じ射$k\rightarrow n$が合成できる. よって射は結合法則を満たす.
        \item $m\rightarrow n$ならば, $m\leqq m$より$m\rightarrow m$が存在し, $m\rightarrow m\rightarrow n$と$m\rightarrow n\rightarrow n$から同じ射$m\rightarrow n$が合成できる. よって, 射は単位法則を満たす.\\
\end{enumerate}
以上より, $C$は圏と定義できる.
\end{exam}
\begin{dfn}
    関手(functor)は, 対象と射についての射で構成し, 圏間をつなぐ射. 圏$X$から圏$Y$への射$F$は,以下の条件を満たすとき, 関手と呼ぶ.
    \begin{itemize}
        \item 対象:$X$の各対象$A$を, $Y$の任意の対象に対応させる. ($A\longmapsto F(A)$)
        \item 射:$X$の各射$f:A\rightarrow B$を, $Y$の任意の射$F(f): F(A)\rightarrow F(B)$に対応させる. ($f\longmapsto F(f)$) さらに以下の性質を満たす.
    \end{itemize}
    \begin{itemize}
        \item[(1)] $X$の任意の射, $f:A\rightarrow B, g:B\rightarrow C$に対して,$F(g\circ f) = F(g)\circ F(f)$
        \item[(2)] $X$の各恒等射, $1_A$に対して, $F(1_A) = 1_F(A)$
    \end{itemize}
    
\end{dfn}

\begin{equation}
    \begin{array}{llll}
         F:& X & \longrightarrow & Y \\
        & \rotatebox{90}{$\in$} & & \rotatebox{90}{$\in$} \\
        \mbox{対象}:& A & \longmapsto & F(A)\\
         \mbox{ 射}:& f & \longmapsto & F(f)\\
    \end{array}
\end{equation}
\begin{hsk}
    関手とは, 圏と圏の間の射である. 準同型写像がその構造を保つように, 関手は圏としての構造を保つ射である. 対象と射にはそれぞれ必ず写し先があり, さらに合成が保たれる.\\
\end{hsk}
\begin{exam}
    圏$P$を, 対象が1,2,...の自然数全体, 射が$m\leqq n$ならば$m \rightarrow n$の圏とする. 次に圏$\SimpComp$を, 対象が単体複体, 射が単体写像の圏とする. では$P$からSimpCompへの射を考える. $P$の各対象を$\SimpComp$の任意の対象へ対応させ, 各射も同様に対応させる. この時, 以下の性質を満たすものを関手$F:P\rightarrow \SimpComp$として扱う.
    \begin{itemize}
        \item $P$の任意の射, $i\stackrel{f}{\to} j\stackrel{g}{\to} k$に対して, $F(g\circ f) = F(g)\circ F(f)$
        \item $P$の各恒等射, $1_i:i\rightarrow i$に対して, $F(1_i) = 1_F(i)$\\
    \end{itemize}
\end{exam}

\begin{dfn}
    自然変換(natural transformation)は, 関手で写された対象同士をつなぐ射からなる, 関手間の射. 圏$X$から圏$Y$への関手$F,G$に対し, $F$から$G$への射$\alpha$は以下の条件を満たすとき, 自然変換と呼ぶ.
    \begin{itemize}
        \item $X$の各対象$A$に対し, $F(A)$から$G(A)$へ射を走らせる ($F(A)\xrightarrow[]{\alpha_A} G(A)$)
        \item $X$の任意の射$A\stackrel{f}{\to} B$に対して, $G(A)\circ \alpha_A = \alpha_B\circ F(A)$
    \end{itemize}
   
\end{dfn}

\begin{equation}
\begin{tikzcd}
A
\arrow[r, bend left, "F"]
\arrow[r, bend right, swap, "G"]
\arrow[r, phantom, bend left, shift right=0.2ex, ""{name=U}]
\arrow[r, phantom, bend right, shift left=0.2ex, swap, ""{name=D}]
&B
\arrow[Rightarrow, from=U, to=D, "\alpha"]
\end{tikzcd}
     \begin{tikzcd}
        F(A)\arrow{r}{F(f)} \arrow{d}{\alpha_A} & F(B)\arrow{d}{\alpha_B}\\
        G(A)\arrow{r}{G(f)} & G(B)\\
    \end{tikzcd}
\end{equation}

\begin{hsk}
    自然変換とは関手と関手の間の射である. 関手が圏の構造を保ったように, 自然変換は関手の構造を保つ. 
\end{hsk}

\noindent\\
抽象単体複体\\
単体写像\\
写像ならば射の法則は満たす?\\
部分圏の定義?\\
次数付き$Z_2[x]加群$\\
$Z_2[x]$が体であり、$\Vect_{Z_2[x]}$がベクトル空間であること?\\
忠実充満\\



\section{結果 Result}
\noindent
$P$とは以下のように定義する圏である.\\
対象: 1,2,...,nの自然数  \\
射: $i \leq j$ ならば,  $i\rightarrow j$ なる射がただ一つ存在. \\
例 2.3.より, 射は定義2.1.(1)~(3)の法則を満たす. 以上から$P$を圏として定義する. \\
$\SimpComp$(抽象単体複体の圏)とは以下のように定義する. \\
対象:単体複体 (ex. $K, L$)\\
射:単体写像   (ex. $K\rightarrow L$)\\
(1)合成法則は, 写像の合成法則と同等のため満たされる. (2)結合法則についても同様. (3)単位法則は, 恒等写像と同等のため満たされる. 以上から$\SimpComp$を圏として定義する.\\
$\Vect$(ベクトル空間の圏)とは対象がベクトル空間, 射が線形写像の圏である. 各法則については上記の$\SimpComp$同様に確認できる. 
以上より$\Vect$を圏として定義する. \\
\noindent\\
次に関手を考える.\\
$P\rightarrow \SimpComp$とは, $P$から$\SimpComp$へ対象と射を対応させ, 定義2.4.(1)(2)を満たすものを関手とする. (例2.6.参照) \\
同様に$P \rightarrow \Vect$の関手も定義する. "この関手の意味の追加"\\
\noindent\\
ここから関手を対象とした圏へ拡張する. \\
$\Fun{P}{\SimpComp}$とは以下のように定義する圏である.\\
対象:$P$から$\SimpComp$への関手\\
射:自然変換\\
自然変換は, 圏$\SimpComp$の射で構成されるため定義2.1.より(1)~(3)の法則は満たされる. 以上から$\Fun{P}{\SimpComp}$を圏として定義する. \\
同様に$\Fun{P}{\Vect}$も圏として定義する. \\
\noindent\\
この二つの圏の間の関手を考える. \\
関手$\Fun{P}{\SimpComp}\rightarrow \Fun{P}{\Vect}$とは,対象と射を対応させ定義2.4.(1)(2)を満たすものである."この関手の意味の追加"\\
\noindent\\
さらに, $\SimpComp$からフィルトレーションのみを抽出するため, 射を包含写像に限定した$\SimpCompInc$を定義する. $\SimpCompInc$とは対象が単体複体, 射が包含写像の圏である. 定義2.1.(1)~(3)の法則に関しては, 射が写像であるため前述の通り満たすとする. ここで, 圏$\SimpCompInc$の任意の射$A\xrightarrow[]{f} B$について$A, B, f\in \SimpComp$であることから, $\SimpCompInc\subset \SimpComp$といえる. 同様に $\Fun{P}{\SimpCompInc}\subset \Fun{P}{\SimpComp}$.\\
\noindent\\
この$\Fun{P}{\SimpCompInc}$から$\Fun{P}{\Vect}$への関手を考えることで, 単体複体フィルトレーションのパーシステントホモロジーを計算できる. "図の作成、まとめの文言、この先の展開"\\
\noindent\\
$\grmodZ$とは以下のように定義する圏である.\\
対象:次数付き$\mathbb{Z}_2[x]$加群\\
射:次数付き$\mathbb{Z}_2[x]$準同型写像\\
法則について, 今回も写像を利用し詳細は割愛する.以上から$\grmodZ$は圏として定義する. 
\noindent\\
$\Vect_{\mathbb{Z}_2}$とは対象が係数$\mathbb{Z}_2$のベクトル空間, 射が線形写像の圏である. $\Vect$では係数について言及しなかったが, 今回は係数が$\mathbb{Z}_2$であることを留意したい.\\
また, 前回同様$\Fun{P}{Vect_{\mathbb{Z}_2}}$も圏として定義する.\\
\noindent\\
ここで$\Fun{P}{Vect_{\mathbb{Z}_2}}$から$\grmodZ$への関手を考える.\\
定義2.4.を満たした関手は忠実充満であるだろうか.\\
忠実充満??






\noindent\\
4.3が忠実充満関手であることを証明する。\\
5.ベクトル空間と次数付き加群、のホモロジー群は同値であると明言する。\\
\end{document}

