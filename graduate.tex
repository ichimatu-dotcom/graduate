\documentclass{article}
\usepackage{graphicx} % Required for inserting images
\usepackage{amsmath}
\usepackage{amsthm}
\usepackage{amssymb}
\usepackage[margin=20truemm]{geometry}
\usepackage{graphicx}
\usepackage{enumerate}
 \theoremstyle{definition}
    \newtheorem{dfn}{定義}[section]
    \newtheorem{prop}[dfn]{命題}
    \newtheorem{lem}[dfn]{補題}
    \newtheorem{thm}[dfn]{定理}
    \newtheorem{cor}[dfn]{系}
    \newtheorem{exam}[dfn]{例}
    \newtheorem{rem}[dfn]{注意}
    \newtheorem{hsk}[dfn]{補足}
    \renewcommand\proofname{\bf 証明}

    \renewcommand{\qedsymbol}{$\blacksquare$}

\title{graduate book}
\author{猪原 盛寿}
\date{October 19th 2023}


\begin{document}
\Large
\maketitle

\section{導入 Introduction}
良心に恥じぬということだけが、我々の確かな報酬である。\\
---セオドア・ソレンソン
\section{背景 background}
圏(対象、射、合成、結合法則、単位法則)\\
\begin{dfn}
圏(category)は,対象と射で構成する. 射は, 対象間に走る矢印. さらに次の三つの法則を満たす.
\begin{enumerate}[(1)]
    \item 合成法則. 任意の対象$A, B, C$に対し, $A\rightarrow B, B\rightarrow C$の任意の射$f, g$は合成と呼ばれる演算, $g \circ f: A\rightarrow C$が成立する.
    \item 結合法則. 任意の対象$A, B, C, D$に対し, $A\rightarrow B, B\rightarrow C, C\rightarrow D$の任意の射$f, g, h$は次の等式を満たす. $(h \circ g) \circ f = h \circ (g \circ f)$  
    \item 単位法則. 任意の対象$A$に対し, 恒等射と呼ばれる射$1_A:A\rightarrow A$が存在し, 次の等式を満たす. (ただし, $B$は任意の対象, $f$は$A\rightarrow B$の任意の射.) $f\circ 1_A = 1_B\circ f = f$  
\end{enumerate}
\end{dfn}
\begin{hsk}
まとめると圏とは, 対象と射からなり, 射が三つの法則を満たすものである.
対象とは, 集合でいうところの要素のことである. しかし、対象に条件は存在しない. 物体でも現象でも, 何でも記号を記号を付けて対象とすることができる. A, ☆, 1, 〇, ...など. 
射とは, 対象間に走る矢印のことである.対象同士を結ぶこと, 向きが存在すること, が条件である. $A \rightarrow B$, 〇$\rightarrow$☆, など. このときのA, 〇を始域といい, 反対にB, ☆は終域という.
合成法則により, 仲介する対象を書かなくとも一本の射で表現できる. 結合法則により, 三つ以上の射の合成は合成する順番を考慮しなくてよい. 単位法則により\\
\end{hsk}
\begin{exam}
対象を1,2,...の自然数全体, 射を$m\leqq n$ならば$m \rightarrow n$ ($m, n \in \mathbb{N}$)とする集まり$C$を考える. この集まりが(1)~(3)の法則を満たすか確認する. 
 \begin{enumerate}[(1)]
        \item 任意の自然数$i, j, k$に対し, $i\leqq j\leqq k$ならば$i\leqq k$より, $i\rightarrow j\rightarrow k$ならば$i\rightarrow k$となる射が存在する.
        \item 任意の自然数$i, j, k, l$に対し$i\leqq j\leqq k\leqq l$のとき, (1)より$i\rightarrow k$, $j \rightarrow l$なる射が存在し, さらに二つの射からそれぞれ同じ$i\rightarrow l$が導ける.
        \item 任意の自然数$i, j$に対し$i\leqq j$のとき$i\rightarrow j$の射が存在する. さらに$i\leqq i$, $j\leqq j$より$i\rightarrow i$, $j\rightarrow j$が存在し, (1)より二つの射からそれぞれ同じ$i\rightarrow j$が導ける.
\end{enumerate}

\end{exam}
圏の例\\
抽象単体複体\\
単体写像\\
写像の合成、から導ける結合法則\\
関手\\


\section{結果 Result}
$P$(順序集合の圏) とは以下のように定義する圏のことである。\\
対象: 1,2,...,nの自然数  \\
射: $i \leq j$ ならば,  $i\rightarrow j$ なる射がただ一つ存在。 \\
$i\rightarrow j \rightarrow k$ ならば,  $i\rightarrow k$ なる射が存在する。そのため、射の合成が定義できる。($i \leq j \leq k$ ならば,  $i \leq k$ より)\\
任意の三つの射$f,g,h$ (1)は等式()を満たす。   $f \circ  (g \circ h) = (f \circ g) \circ h$ 
(右辺左辺どちらも同じ$f\circ g\circ h$を指すため等しい。)これによりPの射は結合法則を満たす。\\
任意のiに対し存在する射$1_i:i\rightarrow i$は次の等式を満たす。$f\circ 1_i = 1_j\circ f = f$これにより単位法則も満たす。
以上から、$P$は圏として定義することが可能となる。\\
\begin{equation}
  \circlearrowleft i\xrightarrow[f]{} j\xrightarrow[g]{} k\xrightarrow[h]{} l 
\end{equation}
\\
次に、$SimpComp$(単体複体の圏)を定義する。\\
対象:単体複体 (ex. $K, L$)\\
射:単体写像   (ex. $K\rightarrow L$)\\
写像の合成より、射の合成は定義できる。\\
結合法則も同様に成り立つ。恒等写像は単体写像の一つなので、恒等射も存在し単位法則は満たされる。\\
以上より、$SimpComp$は圏として定義できる。\\
\\さらに、$Vect$(ベクトル空間の圏)を定義する。\\
対象:ベクトル空間\\
射:線形写像\\
$SimpComp$と同様に、射が写像であることを利用し他の条件を満たす。\\
以上より$Vect$を圏として定義する。\\

では,関手を考える。\\
$$
\begin{array}{llll}
  F:& P & \longrightarrow & SimpComp \\
 & \rotatebox{90}{$\in$} & & \rotatebox{90}{$\in$} \\
\mbox{対象}:& i & \longmapsto & F(i)\\
\mbox{ 射}:& f & \longmapsto & F(f)
\end{array}
$$
上記のように、$P$から$SimpComp$へ対象と射を対応させ、下記の二つの式を満たすものを関手$F$とする。\\
$$
\begin{array}{ll}
 F(f'\circ f) = F(f') \circ F(f)\\
 F(1_i) = 1_{F(i)}
\end{array}
$$\\
同様に、$P \rightarrow Vect$の関手を定義する。\\

ここから、関手圏へ拡張する。\\
$[P \quad SimpComp]$\\
対象:$P$から$SimpComp$への関手\\
射:自然変換\\
”合成、結合法則、単位法則の確認”\\
$K \rightarrow L$と$L \rightarrow M$のとき、$K\rightarrow M$が存在する。\\


$[P \quad Vect]$も同様に行う。\\

ここで、$[P \quad SimpComp]$から、$[P \quad Vect]$への関手を考える。\\
"関手の確認"\\


さらに、$SimpComp$からフィルトレーションのみを抽出するため、射を包含写像のみに限定した$SimpCompInc$を定義したい。$SimpCompInc \subset SimpComp$\\
"部分圏の確認"\\


この$[P\quad SimpCompInc]$から$[P \quad Vect]$へ関手を考えることで、単体複体フィルトレーションのパーシステントホモロジーを計算できる。\\
"関手の確認"

\end{document}
