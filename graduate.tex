\documentclass[a4paper]{jsarticle}
\usepackage[dvipdfmx]{graphicx} % Required for inserting images
\usepackage{amsmath}
\usepackage{amsthm}
\usepackage{amssymb}
\usepackage[margin=20truemm]{geometry}
\usepackage{graphicx}
\usepackage{enumerate}
\usepackage{tikz-cd}
 \theoremstyle{definition}
    \newtheorem{dfn}{定義}[section]
    \newtheorem{prop}[dfn]{命題}
    \newtheorem{lem}[dfn]{補題}
    \newtheorem{thm}[dfn]{定理}
    \newtheorem{cor}[dfn]{系}
    \newtheorem{exam}[dfn]{例}
    \newtheorem{rem}[dfn]{注意}
    \newtheorem{hsk}[dfn]{補足}
    \renewcommand\proofname{\bf 証明}

    \renewcommand{\qedsymbol}{$\blacksquare$}
\newcommand{\SimpComp}{{\mathrm{SimpComp}}}
\newcommand{\Fun}[2]{[#1,~#2]}
\newcommand{\Vect}{{\mathrm{Vect}}}
\newcommand{\SimpCompInc}{{\mathrm{SimpCompInc}}}
\newcommand{\grmodZ}{{\mathrm{grmod \mathbb{Z}_2[x]}}}
\newcommand{\Hom}{{\mathrm{Hom}}}
\newcommand{\ob}{{\mathrm{ob}}}


\title{graduate book}
\author{猪原 盛寿}
\date{October 19th 2023}


\begin{document}
\Large
\maketitle

\section{導入 Introduction}
良心に恥じぬということだけが、我々の確かな報酬である。\\
---セオドア・ソレンソン\\
勇敢にもこの論文を手に取った全ての人に感謝を込めて
\section{背景 background}
数学には, 「関係性」に着目した極めて抽象的な分野が存在する. それは圏論である. 本稿では圏論の基礎的な三つの道具(圏, 関手, 自然変換)を定義し, 第3章へとつなげたい. 手始めに, 圏(category)の定義を確認する.
\begin{dfn}
圏$X$とは以下の条件を満たす数学的構造である. 
\begin{itemize}
    \item 対象(object)の集まり ob($X$)
    \item 射(morphism)の集まり Hom($A,B$)
    (各対象$A, B$に対して$A$から$B$への射の集まりをHom($A, B$)とする)
    \item 射の合成と呼ばれる関数$(\circ)$
\end{itemize}
\begin{equation}
    \begin{array}{lllll}
     \circ &: \Hom (B, C) &\times \Hom (A, B) &\rightarrow &\Hom (A, C)  \\
         &    & (g,f) & \mapsto&  g\circ f
\end{array}
\end{equation}
から構成され, 射は次の二つの法則を満たす.
\begin{enumerate}[(1)]
    \item 結合法則. 任意の射$f\in \Hom(A,B), g\in \Hom(B,C), h\in \Hom(C,D),$は次の等式を満たす. $(h \circ g) \circ f = h \circ (g \circ f)$  
    \item 単位法則. 任意の対象$A$に対し, 恒等射と呼ばれる射$1_A:A\rightarrow A$が存在し, 次の等式を満たす.  $f\circ 1_A = 1_B\circ f = f$  
\end{enumerate}
\begin{equation}
    \begin{tikzcd}
    A\ar{r}{f} \ar[loop left,"1_A"] \ar[bend right=30,swap]{rr}{g\circ f}& B\ar{r}{g} \ar[bend left=30]{rr}{h\circ g}& C\ar{r}{h} & D
\end{tikzcd}
\end{equation}
\end{dfn}

\begin{hsk}
まとめると圏とは, 対象, 射, 射の合成からなり, 射が二つの法則を満たすものである.
対象とは, 集合でいうところの要素のこと. A, ☆, 1, 〇, ...など集合や数字に限らない.
射とは, 対象間に走る矢印のこと.対象同士を結ぶこと, 向きが存在すること, が条件となる. $A \rightarrow B$, 〇$\rightarrow$☆, などで$A$と$B$の間の射は一本とは限らない. このときのA, 〇を始域といい, 反対にB, ☆は終域という. 圏論が「関係性」の数学と呼ばれるように, 圏では射が主役となる. 構成要素の一つに射の合成が入っている. 片方の終域ともう片方の始域が一致するとき, 二つを合成した射が一つ存在しなければならない. 圏$X$が$A\xrightarrow[]{a} B\xrightarrow[]{b} C$かつ, $A\xrightarrow[]{c} C$のとき, $c=b\circ a$である. 結合法則や単位法則は, 写像のそれと同様である. 
\\
\end{hsk}

\begin{exam}
対象を1,2,...の自然数全体, 射を$m\leqq n$ならば$m \rightarrow n$ ($m, n \in \mathbb{N}$)とする集まり$C$を考える.\\ ob($C$)$= \mathbb{N}$. $m\leqq n$のとき$\Hom (m, n)=\{m\rightarrow n\}$. $m>n$のとき$\Hom (m, n)=\{\}$. $l, m, n\in$ ob$(X)$, $l \rightarrow m \rightarrow n$ならば, $l\leqq m\leqq n$から$l\leqq n$より$l \rightarrow n$. よって射の合成は含まれている. 次に射が二つの法則を満たすか確認する. 
 \begin{enumerate}[(1)]
        \item $k\rightarrow l\rightarrow m\rightarrow n$ならば, 合成により$k\rightarrow m$, $l \rightarrow n$が存在し, さらに二つの射を利用し同じ射$k\rightarrow n$が合成できる. よって射は結合法則を満たす.
        \item $m\rightarrow n$ならば, $m\leqq m$より$m\rightarrow m$が存在し, $m\rightarrow m\rightarrow n$と$m\rightarrow n\rightarrow n$から同じ射$m\rightarrow n$が合成できる. よって, 射は単位法則を満たす.
\end{enumerate}
以上より, $C$は圏と定義できる.\\
\end{exam}
圏を定義出来たおかげで,対象と対象の関係性が射によって表せられるようになった. では圏と圏の関係性を表すときはどうすればよいだろうか?答えはシンプルで, もう一度射を伸ばせばよい. 圏論ではその射を, 「関手(functor)」と呼ぶ.
\begin{dfn}
    $X, Y$を圏とする. 関手$F:X\rightarrow Y$は, 
    \begin{itemize}
        \item 対象についての関数:ob$(X)\rightarrow$ ob$(Y)$  $[A\mapsto F(A)]$
        \item 射についての関数:$\Hom (A, B)\rightarrow \Hom (F(A), F(B))$  $[f\mapsto F(f)]$
    \end{itemize}
からなり, 射についての関数が次の二つの公理を満たす.
    \begin{itemize}
        \item[(1)] 圏$X$で, $A\xrightarrow[]{f} B\xrightarrow[]{g} C$のとき, $F(g\circ f) = F(g)\circ F(f)$
        \item[(2)] $A\in$ob$(X)$のとき, $F(1_A) = 1_F(A)$
    \end{itemize}
\begin{equation}
    \begin{array}{llll}
         F:& X & \longrightarrow & Y \\
        & \rotatebox{90}{$\in$} & & \rotatebox{90}{$\in$} \\
        \mbox{対象}:& A & \longmapsto & F(A)\\
         \mbox{ 射}:& f & \longmapsto & F(f)\\
    \end{array}
\end{equation}
\end{dfn}

\begin{hsk}
    関手とは, 圏と圏の間の射である. 対象と射についての関数(射)で構成し, (1)で合成を(2)で単位法則を保存する. 準同型写像がその構造を保つように, 関手は圏としての構造を保つ射である. 以降は, 記号や矢印が増えるためメモ等で図を書くことをお願いしたい.\\
\end{hsk}
\begin{exam}
    圏$C$を, 対象が1,2,...の自然数全体, 射が$m\leqq n$ならば$m \rightarrow n$の圏とする. 次に圏$D$を, 対象が単体複体, 射が単体写像の圏とする. では$C$から$D$への関手$F$を考える. $l, m, n\in\ob (C), $ $ L, M, N\in\ob(D)$とすると,\\ 
    $\Hom (m, n)\rightarrow \Hom (F(m), F(n))$: $\Hom (m, n)=\{f_m\}$, $F(m)=M, F(n)=N$, $\Hom (M, N)=\{g, h\}$ならば, $F(f_m)=g$または$F(f_m)=h$となる必要がある.\\
    次に射についての関数が次の二つの法則を満たすか確認する.\\
    $F(l)=L, \Hom (L, M)=\{i,j\}$とすると, 
    
    \begin{itemize}
        \item[(1)] 圏$C$で, $l\stackrel{f_l}{\to} m\stackrel{f_m}{\to} n$ならば, $F(f_m\circ f_l) =i かつ F(f_l)\circ F(f_m) = i, またはF(f_m\circ f_l) =j かつ F(f_l)\circ F(f_m) = j$のときのみ, この公理を満たす.
        \item [(2)] $m\in$ob$(C)$で, $1_m:m\rightarrow m$ならば, $F(1_m) = 1_F(m)(=1_M)$のときのみ, この公理を満たす.
    \end{itemize}
    以上から, いくつかの条件を満たすとき$F$を関手と定義できる.\\
\end{exam}
関手のおかげで圏と圏の関係性を表せられるようになった. では関手と関手の関係性を表すときはどうすればよいだろうか?察しの良い読者はもうお気づきだろうが答えはやはりシンプルである. もう一度だけ射を伸ばそう, 圏論ではその射を「自然変換(natural transformation)」と呼ぶ.
\begin{dfn}
    $X, Y$を圏, $F, G$を$X$から$Y$への関手, とする.\\
    $F$から$G$への自然変換$\alpha$とは,
    \begin{itemize}
        \item $Y$の射の族 ($F(A)\xrightarrow[]{\alpha_A} G(A)$, $A\in\ob(X)$)
    \end{itemize}
    からなり, この射の族が次の一つの等式を満たす.
    \begin{itemize}
        \item $A, B\in\ob(X), A\xrightarrow[]{f}B$ならば, $G(f)\circ \alpha_A = \alpha_B\circ F(f)$
    \end{itemize}
\end{dfn}
\begin{equation}
\begin{tikzcd}
X
\arrow[r, bend left, "F"]
\arrow[r, bend right, swap, "G"]
\arrow[r, phantom, bend left, shift right=0.2ex, ""{name=U}]
\arrow[r, phantom, bend right, shift left=0.2ex, swap, ""{name=D}]
&Y
\arrow[Rightarrow, from=U, to=D, "\alpha"]
\end{tikzcd}
     \begin{tikzcd}
        F(A)\arrow{r}{F(f)} \arrow{d}{\alpha_A} & F(B)\arrow{d}{\alpha_B}\\
        G(A)\arrow{r}{G(f)} & G(B)
    \end{tikzcd}
\end{equation}
\begin{hsk}
    自然変換とは関手と関手の間の射である. 対象についての関数で写された対象同士を結ぶ射で構成する. よってその射の全ては, 関手の終域の圏に帰属する. 満たされるべきただ一つの等式は, 射についての関数で写された射と自然変換の射について条件を課すものである. 上記右側の図において, $F(A)\rightarrow G(B)$について二通りの合成が可能となる. その二つの合成射が一致する必要が自然変換にはある.
\end{hsk}
\begin{exam}
    
\end{exam}
\begin{dfn}
    $f:K\rightarrow K'$が単体写像とは, $K$の各頂点を$K'$のある頂点に写し次の条件を満たすもののことである. 
\begin{equation}
    \forall\sigma =\{a_0,...,a_q\}\in K\Rightarrow \{f(a_0),...,f(a_q)\}\in K'
\end{equation}
\end{dfn}
\begin{hsk}
    ここで, 単体写像は単体を写すのではなく頂点のみを写すことに注意したい. 仮に単体そのものを写す写像を単体写像'とすると, 抽象単体複体の構造を保つことなく写像は機能を失ってしまう. また, 単体の次元の順序関係を保つ条件として, $\sigma\subseteq\tau \Rightarrow f(\sigma) \subseteq f(\tau)$を加えても次元の順序関係のみが保たれ, 単体としては遥か上の次元へと写すことが可能となってしまう. 結果的に, 抽象単体複体の構造をうまく保つ写像を作ろうとすれば, 上記のように頂点のみを写し条件として単体が保存されることを加えた定義になるのである.
\end{hsk}
\begin{dfn}
    次数付き$\mathbb{Z}_2[x]$加群$M$とは, $\mathbb{Z}_2[x]$加群である$M$にアーベル群としての直和分解を与え, $x^jM_i\subset M_{i+j}$を満たすものである($i,j\in\mathbb{N}_0$).
    \begin{equation}
        M=\bigoplus_{i=0}^{\infty} M_i\notag
    \end{equation}
\end{dfn}
\begin{dfn}
    次数付き$\mathbb{Z}_2[x]$準同型写像とは, $M, N$を次数付き$\mathbb{Z}_2[x]$加群とすると, $f:M\rightarrow N$が$\mathbb{Z}_2[x]$準同型写像であって, $f(M_i)\subset N_i$を満たすものである.
\end{dfn}
\begin{dfn}
    $\mathbb{Z}_2[x]$準同型写像とは, $M, N$を$\mathbb{Z}_2[x]$加群とすると, $f:M\rightarrow N$が次の条件を満たすものである. $(a, b\in M)$
        \begin{flalign}
             &(1) f(a+b) = f(a)+f(b)&\\
             &(2) f(xa)= xf(a)& \notag
        \end{flalign}
\end{dfn}
\begin{dfn}
    抽象単体複体とは, 有限集合$V$と$V$の部分集合の集まり$\Sigma$の組($V,\Sigma$)で以下の条件を満たすもののことである.\\
    \noindent
    (1)$\forall v\in V$ならば, $\{v\}\in\Sigma$ \\
    (2)$\forall a\in \Sigma, b\subset a$ならば, $b\in \Sigma$
\end{dfn}

\noindent\\
$\mathbb{R}^N$内の有限個のデータXが部分集合の集まり$\Phi=\{B_i\subset \mathbb{R}^N| 
 i=1,...,m\}$に被覆されている状況を考える. 式で書くと以下のようになる.
 \begin{equation}
     X = \bigcup_{i=1}^{m}B_i \notag
 \end{equation}
このとき, 頂点集合を$V=\{1,...,m\}$, 単体の集まり$\Sigma$を, 
\begin{equation}
   \Sigma = \left\{ \{1,...,k\} \middle| \bigcap_{i=1}^{k}B_{i} \neq \emptyset \right\} 
 \notag
 \end{equation}
とすると($\emptyset$ は空集合の意), これは抽象単体複体となる. この抽象単体複体を$\Phi$の脈体と呼び, $\mathcal{N}(\Phi)$で表す. 
\begin{thm}
    (脈体定理) $X\subset \mathbb{R}^N$が凸閉集合の有限個の集まり$\Phi=\{B_i\subset \mathbb{R}^N| i=1,...,m\}$に被覆されているとき, $X$と$\mathcal{N}(\Phi)$はホモトピー同値となる.
\end{thm}
以上よりデータからホモトピー不変な量を調べる時は, 定理の仮定を満たす$X$を作り脈体に変換してから調べてよいこととなる. 一般的に, 脈体はもとのデータより扱いやすく特に計算機との相性が良い. 脈体定理の証明は[ここに参考文献]を参照されたい. \\



忠実充満\\



\section{結果 Result}
$P$とは次のように定義する圏である.対象が1,2,...,nの自然数, 射が$i \leq j$ ならば,  $i\rightarrow j(i, j \in \mathbb{N})$とする. 前述の通り, 射の合成の存在と射が満たすべき二つの法則は満たされる. 以上から$P$を圏として定義する. \\
$\SimpComp$とは次のように定義する. 対象が抽象単体複体, 射が単体写像とする. 
単体写像と単体写像の合成が, 単体写像になるか確認する. 単体写像の定義から, $f:K\rightarrow K'$, $f':K'\rightarrow K''$を単体写像とすると$\forall\sigma =\{a_0,...,a_q\}\in K\Rightarrow \{f(a_0),...,f(a_q)\}\in K'\Rightarrow\{f'(f(a_0)),...,f'(f(a_q))\}\in K''\Rightarrow\{f'\circ f(a_0),...,f'\circ f(a_q)\}\in K''$となるため, これは定義を満たし, $f'\circ f:K\rightarrow K''$も単体写像となる. 次に$K\xrightarrow[]{f} K'\xrightarrow[]{g} K''\xrightarrow[]{h}K'''$のとき, $K$の任意の頂点$a_i$について$(h\circ g)\circ f(a_i)=(h\circ g)(f(a_i))=h\circ g\circ f(a_i)$かつ, $h\circ (g\circ f)(a_i)=h(g\circ f(a_i))=h\circ g\circ f(a_i)$より, $(h\circ g)\circ f=h\circ (g\circ f)$となって結合法則が満たされる. また$1_K:K\rightarrow K, 1_K':K'\rightarrow K'$のとき, $f\circ 1_K(a_i)=f(a_i)$かつ$1_{K'}\circ f(a_i)=1_{K'}(f(a_i))=f(a_i)$より, $f\circ 1_K=1_{K'}\circ f=f$となって単位法則も満たされた. 以上より, SimpCompは圏として定義する.\\
$\Vect$とは対象がベクトル空間, 射が線形写像の圏である. 線形写像の合成が線形写像になるか確認する. $f:V\rightarrow V'$, $f':V'\rightarrow V''$を線形写像とし, $a, b$をスカラー, $x, y\in V$とすると線形写像の定義から, $f'\circ f(ax+by)=f'(a(x)+bf(y))=af'(f(x))+bf'(f(y))=a(f'\circ f)(x)+b(f'\circ f)(y)$となるため, これは定義を満たし$f'\circ f:V\rightarrow V''$も線形写像となる. 結合法則, 単位法則については上と同様. 以上よりVectは圏として定義する. \\
$\SimpComp$から$\Vect$への関手を1つ具体的に定義する. $H_q:\SimpComp \rightarrow \Vect$とする. 圏$\SimpComp$の任意の射$f:K\rightarrow K'$ならば, 圏Vectで$H_q(f):H_q(K)\rightarrow H_q(K')$である. $H_q(f)$が関手となるため満たすべき条件は二つ. (1)$K\xrightarrow[]{f}K'\xrightarrow[]{f'}K"$ならば, $H_q(f'\circ f) = H_q(f')\circ H_q(f)$. (2) $1_K:K\rightarrow K$ならば, $H_q(1_K)=1_{H_q(K)}$. 今, $H_q(f)$を$K$の単体$z$を用いて, 
\begin{equation}
    \begin{array}{llll}
         H_q(f):& H_q(K) & \longrightarrow & H_q(K') \\
        & \rotatebox{90}{$\in$} & & \rotatebox{90}{$\in$} \\
        & [z] & \longmapsto & [f_q(z)]\\
    \end{array}
\end{equation}
と定義する. $K\xrightarrow[]{f}K'\xrightarrow[]{f'}K"$のとき, $H_q(f')\circ H_q(f)([z]) = H_q(f')([f_q(z)]) = [f'_q\circ f_q(z)]$となる. 一方$H_q(f'\circ f) ([z])  = [ (f'\circ f)_q(z)]$となるため, (1)を示すためには$f'_q\circ f_q=(f'\circ f)_q$の確認が急務となる. $f_q$は単体写像が誘導するチェイン群の準同型写像. $q$次元の鎖群$C_q(K)$を用いて $f_q:C_q(K)  \longrightarrow C_q(K')$と表し,
\begin{equation}
    f_q(\langle a_0...a_q\rangle)=\left\{
    \begin{array}{c l}	
    \langle f(a_0)...f(a_q)\rangle & (f(a_0),...,f(a_q)が相異なるとき)\\
    0 & (そうでないとき)
\end{array}\right.
\end{equation}
と定義する. $f'_q\circ f_q(\langle a_0...a_q\rangle)$は$f(a_0),...,f(a_q)$が相異なるとき$f'_q(\langle f(a_0)...f(a_q)\rangle)$で, そうでないとき$f'_q(0)$である. さらに$f'_q(\langle f(a_0)...f(a_q)\rangle)$は, $f'\circ f(a_0),...,f'\circ f(a_q)$が相異なるとき$\langle f'\circ f(a_0)...f'\circ f(a_q)\rangle$で, そうでないとき0である. また$f$は単体写像であるため$f'_q(0)=f(0)=0$となる. よって, $f'_q\circ f_q(\langle a_0...a_q\rangle)$は$f'\circ f(a_0),...,f'\circ f(a_q)$が相異なるとき$\langle f'\circ f(a_0)...f'\circ f(a_q)\rangle$で, そうでないとき0と書くことができる. そうしてこれは, $(f'\circ f)_q(\langle a_0...a_q\rangle)$と等しいため, $f'_q\circ f_q=(f'\circ f)_q$がいえる. よって(1)の$K\xrightarrow[]{f}K'\xrightarrow[]{f'}K"$ならば, $H_q(f'\circ f) = H_q(f')\circ H_q(f)$は満たされた.\\
次に$1_K:K\rightarrow K$のとき, $H_q(1_K):H_q(K)\rightarrow H_q(K)$で$H_q(1_K)([z])=[1_K(z)]=[z]$となる. これは恒等写像の性質を満たすため, $H_q(1_K)=1_{H_q(K)}$がいえる. よって(2)も満たすことができたので, $H_q:\SimpComp\rightarrow \Vect$は関手と定義できる. これをホモロジー関手と呼ぶ.\\
\noindent\\
次に数字を割りあてた単体複体からのホモロジー関手を考える.\\
$P\rightarrow \SimpComp$とは, $P$から$\SimpComp$への関手である. 前述の通り, いくつかの条件を満たし定義される. 同様に$P \rightarrow \Vect$の関手も定義する. ここから関手を対象とした圏へ拡張する. $\Fun{P}{\SimpComp}$とは次のように定義する圏である. 対象が$P$から$\SimpComp$への関手, 射が自然変換とする. 自然変換は, 圏$\SimpComp$の射で構成されるため, 射の合成の存在と射が満たすべき二つの法則は満たされる. 以上から$\Fun{P}{\SimpComp}$を圏として定義する. 同様に$\Fun{P}{\Vect}$も圏として定義する. \\
\noindent\\
この二つの圏の間の関手$H_q:\Fun{P}{\SimpComp}\rightarrow \Fun{P}{\Vect}$は,
\begin{equation}
    \begin{array}{llll}
         H_q:& \Fun{P}{\SimpComp} & \longrightarrow & \Fun{P}{\Vect} \\
        & \rotatebox{90}{$\in$} & & \rotatebox{90}{$\in$} \\
        & \mathbb{K}& \longmapsto & H_q(\mathbb{K})\\
         & =\{K_0\rightarrow ...\rightarrow K_n\} &  & =\{Hq(K_0)\rightarrow ...\rightarrow H_q(K_n)\}\\
    \end{array}
\end{equation}
となる. 前に定義したようにこの関手は単体複体からホモロジーを計算することを表している. 関手圏とすることで, 列ごとの計算に拡張できた. しかし, SimpCompの射は単体写像であるためこのままでは単なる列である. 
そこでさらに, $\SimpComp$からフィルトレーションのみを抽出するため, 射を包含写像に限定した$\SimpCompInc$を定義する. $\SimpCompInc$とは対象が単体複体, 射が包含写像の圏である. 圏SimpCompIncの対象, 射, 射の合成, 恒等射はいつでも圏SimpCompに属するため, その部分圏といえる. 同様に $\Fun{P}{\SimpCompInc}\subset \Fun{P}{\SimpComp}$. これにより$\Fun{P}{\SimpCompInc} \rightarrow \Fun{P}{\Vect}$は単体複体のフィルトレーションからのベクトル空間のパーシステントホモロジーを計算することを表しているといえる. \\

以上で, ベクトル空間でのパーシステントホモロジーの計算を圏論の言葉で書き表すことができた. ここからは, 次数付き$\mathbb{Z}_2[x]$加群でのパーシステントホモロジーの計算を圏論で表現し, その違いや関係性をあらわにしていきたい.\\
\noindent\\
$\grmodZ$とは以下のように定義する圏である. 対象が次数付き$\mathbb{Z}_2[x]$加群, 射が次数付き$\mathbb{Z}_2[x]$準同型写像とする. 次数付き$\mathbb{Z}_2[x]$準同型写像の合成が再び次数付き$\mathbb{Z}_2[x]$準同型写像となるか確認する. 次数付き$\mathbb{Z}_2[x]$準同型写像の定義から,



$\Vect_{\mathbb{Z}_2}$とは対象が係数$\mathbb{Z}_2$のベクトル空間, 射が線形写像の圏である. $\Vect$では係数について言及しなかったが, 今回は係数が$\mathbb{Z}_2$であることを留意したい.\\
また, 前回同様$\Fun{P}{Vect_{\mathbb{Z}_2}}$も圏として定義する.\\
\noindent\\
ここで$\Fun{P}{Vect_{\mathbb{Z}_2}}$から$\grmodZ$への関手を考える.\\
定義2.4.を満たした関手は忠実充満であるだろうか.\\
忠実充満??






\noindent\\
4.3が忠実充満関手であることを証明する。\\
5.ベクトル空間と次数付き加群、のホモロジー群は同値であると明言する。\\
\end{document}

